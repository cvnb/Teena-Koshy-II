\chapter{Summary}
\chaptermark{Summary}

\begin{itemize}

\item[$\blacktriangleright$] An insight into the etiology of congenital heart malformations is directly linked to our understanding of cardiac development. Thus to identify an association between a homogenous group of congenital heart malformations with mutations or variations in selected candidate genes 100 cases with CTHD and 100 age  matched controls were recruited.

\item[$\blacktriangleright$] Since cytogenetic screening of the cases diagnosed with CTHD was the primary step for inclusion in mutation analysis, chromosomal analysis and FISH for the 22q11 microdeletion was performed. While 3\% of the cases showed chromosomal abnormalities, 1\% showed the 22q11 micro- deletion.

\item[$\blacktriangleright$] \textit{TBX1}, a transcription factor of the T-box family is known to have an important role in the regulation of cardiac developmental processes and  was screened for mutations by PCR- Sanger sequencing. Neither the cases nor the control subjects showed any mutations or sequence variants.

\item[$\blacktriangleright$] \textit{NKX2.5} is expressed during early cardiac morphogenesis and functions as a pivotal regulatory protein. While a somatic mutational frequency of 6\% was detected in the cardiac tissue DNA, a mutational frequency of 11\% was observed in the lymphocytic DNA. A marginal but significant fold change in gene expression in the tissue samples relative to the blood samples was also seen.

\item[$\blacktriangleright$] SNP in genes that code for key enzymes in the folate pathway may alter metabolic activity and influence the risk for CTHD. Logistic regression analyses revealed that for the rs1801131 genotypes, subjects carrying the CC variant homozygote had a significant association with the risk of CTHD. For the mothers who did not have periconceptional vitamins there was a significant difference between cases and controls for the CC genotype of \textit{MTHFR}: rs1801131 A>C implying that the absence of sufficient folic acid could increase the risk for CTHD risk in infants with the variant genotype.

\item[$\blacktriangleright$] Promising results were obtained for the NKX2.5 and genes involved in the folate metabolism, providing a starting point for future studies in Indian children with CTHD.

\end{itemize}