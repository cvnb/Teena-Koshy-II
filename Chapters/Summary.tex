\chapter*{Summary}
\chaptermark{Summary}

\begin{itemize}

\item[$\blacktriangleright$] To identify the genetic cause of intellectual disability, study subjects (n=130) were recruited and multiple assays were employed to screen the whole genome.

\item[$\blacktriangleright$] Severe intellectual disability was identified in 12\% of the subjects and 11\% were moderately disabled, based on IQ evaluation. A clinical evaluation of developmental delay was identified in 77\% of subjects.

\item[$\blacktriangleright$] Conventional cytogenetic analysis by high resolution GTG banding is a whole genome approach and revealed that 3\% of study subjects showed chromosomal abnormalities (three deletions and one inversion).

\item[$\blacktriangleright$] PCR-based screening of \textit{FMR1} gene, a characteristic feature of fragile X syndrome, did not detect any mutations in the study population.

\item[$\blacktriangleright$] Subtelomeric rearrangements examined with FISH were detected at a frequency of 7.7\% and two balanced rearrangements were detected in the parental samples.

\item[$\blacktriangleright$] MLPA, used to evaluate interstitial chromosomal rearrangements, yielded a frequency of 2\%. No microdeletions/microduplications were detected using the QMPSF technique.

\item[$\blacktriangleright$] Though targeted and specialized techniques permitted the detection and delineation of more rearrangements, phenotype and genotype correlation was not observed in majority of the study subjects.

\end{itemize}