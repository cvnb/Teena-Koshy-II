\addchaptertocentry{Abstract}
\section*{Abstract}


\clearpage

%\clearpage
%-----------------------------------
%	SECTION 2
%-----------------------------------

The transcriptional and molecular machineries that control heart development are tremendously complex and sensitive to genetic and environmental disturbances. This is reflected by the high incidence of congenital cardiac anomalies in the human population, affecting approximately 1\% of all live births. 

Conotruncal heart defects (CTHD), a relatively homogenous subset of heart defects, originate from an initial development disturbance of the conotruncal septum. This group accounts for 20\% of all congenital heart malformations and include tetralogy of fallot (TOF), pulmonary atresia with ventricular septal defect (PA/VSD), double outlet of right ventricular (DORV), transposition of the great arteries (TGA), persistent truncus arteriosus (PTA) and interrupted aortic arch (IAA). It is postulated that the etiology of CTHD is multifactorial and consists of a combination of environmental and genetic factors that act as disease modifiers, accounting for the wide variation in phenotypic expression and clinical outcomes of these disorders. An understanding of the etiology of CTHD would undoubtedly benefit from comprehensive insight into the genetic and molecular networks that orchestrate cardiac development. Therefore a systematic study, within the context of an Indian population, could provide clues on the genetic basis of CTHD. 

The thesis is divided into six chapters. The first chapter is an introduction with a general overview to CTHD and the various risk factors, with emphasis on the genetic causes. It also summarizes the research that has led to the current paradigm of heart development. The second chapter describes the study design, subject selection, sample collection and the general methodology adopted to perform the work. The exclusion of syndrome associated CTHD was done by karyotyping and Fluorescence in situ hybridization (FISH). Genotyping involved collection of blood sample from the study subjects, isolation of genomic deoxyribonucleic acid (DNA) followed by analysis of gene variations by polymerase chain reaction (PCR) and Sanger sequencing. Ribonucleic acid (RNA) was extracted from cardiac tissue obtained from 55 of the cases and processed for \textit{NKX2.5} expression analysis using real-time PCR.

The third chapter of the thesis is focused on screening for chromosomal rearrangements in children with CTHD.  It has been estimated that between 8 and 13\% of all cases of CHD are associated with chromosomal abnormalities. The detection of chromosome abnormalities was previously limited to either aneuploidies or large rearrangements detectable by classical cytogenetic analysis. In recent years, submicroscopic chromosomal rearrangements have been discovered to play an important role in causing both syndromic and isolated CTHD. One well-studied example is 22q11.2 microdeletion syndrome, which accounts for a significant portion (6\%) of all cases of CTHD. Since cytogenetic screening of individuals diagnosed with CTHD was the primary step in the inclusion of cases for mutation analysis, chromosomal analysis and FISH for 22q11 microdeletion analysis was performed for the cases only. In this study, 3\% of the cases showed chromosomal abnormalities and 1\% of the cases showed the 22q11 micro- deletion.

The fourth chapter deals with \textit{TBX1} gene mutation screening in the cases identified as non-syndromic CTHD. \textit{TBX1}, a transcription factor of the T-box family is known to have an important role in the regulation of developmental processes. Sanger sequencing of the \textit{TBX1} exons revealed that neither the cases nor the control subjects showed any mutations or sequence variants. While the results did not corroborate the role of \textit{TBX1} in the pathogenesis of non-syndromic CTHD, it prompted consideration of other key developmental genes like \textit{NKX2.5}. This was analyzed in chapter five where \textit{NKX2.5} mutations in DNA from both lymphocytes and diseased cardiac tissue were investigated. \textit{NKX2.5} is expressed during early cardiac morphogenesis and functions as a pivotal regulatory protein and was screened for mutations by PCR- Sanger sequencing. While a somatic mutational frequency of 6\% was detected, a mutational frequency of 11\% was observed in lymphocytic DNA. A marginal but significant fold change in gene expression in the tissue samples relative to the blood samples was also seen.

Finally, the sixth chapter explored the association of variants of folate metabolism genes and the risk of non-syndromic CTHD. It has been demonstrated by the functional role of these genes that polymorphisms in genes that code for these key enzymes may alter its activity and influence the risk for CTHD. Based on the strength of evidence of previously published association studies, six genetic variants were chosen for analysis: rs1801133, rs1801131 of methylenetetrahydrofolate reductase (\textit{MTHFR}), rs1051266 of solute carrier family 19 (\textit{SLC19A1}), rs1805087 of methionine synthase (\textit{MTR}) and rs1801394 and rs1532268 of methionine synthase reductase (\textit{MTRR}). Logistic regression analyses revealed that for the rs1801131 genotypes, subjects carrying the CC variant homozygote had a significant association with the risk of CTHD. For the mothers that did not have periconceptional vitamins there was a significant difference between cases and controls for the CC genotype of MTHFR: rs1801131 A>C implying that the absence of sufficient folic acid could increase the risk for CTHD risk in infants with the variant genotype. 

Promising results were obtained for the \textit{NKX2.5} and genes involved in the folate metabolism, providing a possible direction for future studies in CTHD. The significance and implications of the detected abnormalities with relevance to CTHD are discussed in the thesis.
